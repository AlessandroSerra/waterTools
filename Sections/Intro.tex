Water is a paradigmatic hydrogen-bonded system whose macroscopic properties emerge from a highly cooperative and continuously fluctuating network of intermolecular interactions. Unlike simple liquids, its local structure is neither rigid nor fully random: directional hydrogen bonds impose transient constraints that couple intermolecular organization to intramolecular vibrations. As a result, structural correlations and molecular motions are deeply intertwined, and even localized perturbations can induce collective responses across the network. This interplay persists across phases, from the disordered liquid to crystalline ices with distinct topologies and degrees of proton order, making water an ideal framework to investigate how vibrational excitations propagate in a correlated molecular environment.\\

Experimentally, ultrafast vibrational spectroscopy has provided detailed insight into the nonequilibrium dynamics of water. Infrared pump–probe measurements have revealed sub-picosecond relaxation of the O–H stretching mode and rapid redistribution of excess energy toward bending and librational modes within the hydrogen-bond network \cite{dettori2019energy}. Complementary two-dimensional infrared (2D-IR) spectroscopy has further established a direct connection between vibrational frequency and local hydrogen-bond configuration, enabling the time-resolved observation of spectral diffusion and hydrogen-bond rearrangements \cite{lspot}. In particular, the evolution of 2D line shapes demonstrates that the O–H stretching frequency reflects instantaneous hydrogen-bond geometry and that structural fluctuations occur on femtosecond to picosecond timescales. While these techniques provide exquisite temporal resolution and indirect structural sensitivity, disentangling vibrational energy redistribution from the underlying microscopic reorganization of the hydrogen-bond network remains challenging.\\

Computationally, atomistic simulations provide direct access to the microscopic pathways underlying vibrational energy relaxation in hydrogen-bonded liquids. On the one hand, equilibrium molecular dynamics combined with mixed quantum–classical frequency mapping schemes has enabled the calculation of infrared and two-dimensional infrared response functions from atomistic trajectories, establishing a quantitative link between vibrational frequency fluctuations and hydrogen-bond rearrangements \cite{roberts2009structural}. In this framework, simulations serve both as an interpretative tool for spectral diffusion and as a benchmark for water models against experimental observables. On the other hand, nonequilibrium molecular dynamics approaches have been developed to mimic pump–probe excitation protocols directly. Methods based on frequency-selective thermostats allow controlled excitation of a targeted vibrational mode while leaving the remaining degrees of freedom near equilibrium, followed by microcanonical evolution to probe energy redistribution pathways. A notable example is the generalized Langevin equation (GLE) approach introduced to simulate vibrational energy relaxation in pump–probe spectroscopy, where a colored-noise “hotspot” thermostat selectively excites a narrow frequency window and subsequent relaxation is monitored in the NVE ensemble \cite{dettori2017simulating}.\\
While these methods provide molecular-level insight into relaxation mechanisms and characteristic time scales, establishing a direct correlation between selective vibrational excitation and the transient structural response of the hydrogen-bond network remains incomplete, particularly across phases with distinct topologies.\\

In this work, we address this challenge by combining selective vibrational excitation with large-scale atomistic simulations of water across distinct phases. We initialize a controlled nonequilibrium state by exciting a fraction of O–H stretching modes using Wigner sampling, followed by microcanonical molecular dynamics evolution with a machine-learned interatomic potential that retains near first-principles accuracy while enabling system sizes sufficient to capture collective network effects. Rather than directly simulating a spectroscopic observable, our approach focuses on the microscopic dynamics of energy redistribution and structural response under controlled excitation conditions. By analyzing time-resolved vibrational densities of states, mode-resolved temperatures, and radial distribution functions, we directly correlate vibrational energy redistribution with the transient structural response of the hydrogen-bond network. Importantly, we perform a systematic comparison between liquid water, proton-disordered ice Ih, and proton-ordered ice IX, thereby disentangling the roles of structural disorder, lattice constraints, and proton ordering in governing ultrafast energy flow. This unified framework provides an atomistic perspective on how network topology and degree of order influence the coupling between localized vibrational excitation and collective hydrogen-bond rearrangements.
